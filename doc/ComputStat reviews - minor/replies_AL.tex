\documentclass[]{article}

\usepackage[utf8]{inputenc}
\usepackage{amsfonts}
\usepackage{amsmath}
\usepackage{natbib} 
\usepackage{color}
\usepackage{xcolor}
\usepackage{soul}

\usepackage{indentfirst} % indent first paragraph of section
%opening

\newcommand\todo[1]{\textcolor{red}{#1}}
\newcommand\done[1]{\textcolor{red}{#1}}
\newcommand\marie[1]{\textcolor{orange}{#1}}
\newcommand\bert[1]{\textcolor{red}{#1}}
\newcommand\julien[1]{\textcolor{green}{#1}}
\newcommand\alberto[1]{\textcolor{cyan}{#1}}
\newcommand\reply[1]{\textcolor{orange}{#1}}

\title{Clustering over user features and latent behavioral functions with dual-view mixture models \\ \texttt{-Reviews-}}

\author{}

\begin{document}
	\date{}
	\maketitle
	
	Dear Editors,
	
	We sincerely thank you again for your thorough reviews and salient observations. Following your comments and suggestions, we fixed the notation typos and inconsistencies, missing references, plot shapes and the like. We also made some clarifications and added some information following the suggestions of the Associate Editor.
	
	Lastly, we would also like to suggest a small change in the title of the paper to clearly anticipate the non-parametric nature of our main model. The title would change to ``\textit{Non-parametric} clustering over...".   
	
	\bigskip
	\noindent
	\textbf{List of detailed responses:}\\
	(Q: editor comments.
	A: authors replies.)
	\subsection*{Editor-in-Chief}
	
	\vspace{3mm}	
	 \textbf{Q1}: Let me emphasize what the AE already mentioned: "Maybe modify the figures to not only differentiate groups by color, but also with different plot symbols so that they can also be distinguished when printed in gray scale."

Our print issue by default gets published in grayscale (unless you want to pay for color figures). So many of the differently colored points in your figures will become almost indistinguishable for readers of our print issue.
		
		\textcolor{blue}{  \textbf{A1}: We have added different line types and point shapes in the figures.} 
	\vspace{3mm}	

	\textbf{Q2}: Please minimize the use of footnotes and add the text back to the main text, e.g., in parentheses.
		
		\textcolor{blue}{  \textbf{A2}: We have integrated into the text the footnotes of page 8 (about  MLEs), page 11 (Geweke's test), page 14 (users/thread ratio) and page 16 (citation of R package mclust}.

	\vspace{3mm}	
		
\textbf{Q3}:  In footnotes 3 and 6, you refer to the coda and mclust R packages. Formally cite R and R packages you use in the main text and add to the References. Look at a recently published article from our journal for reference.
		
		\textcolor{blue}{  \textbf{A3}: We have added references to both packages.} 

	\vspace{3mm}	

\textbf{Q4}:  Place table captions above the table.

		\textcolor{blue}{  \textbf{A4}: We have placed caption above the table (page 11)} 

	\vspace{3mm}	

\textbf{Q5}: Please check the completeness of your citations, in particular with respect to missing page numbers and missing editor names for conference proceedings.
		
		\textcolor{blue}{  \textbf{A5}: We have added missing pages and editors. Now all conference proceedings have their corresponding pages and editors.} 
	

	\vspace{3mm}
	
	
	\subsection*{Associate Editor}
	\textbf{Q1}: Please consider changing behaviors view to behavior view or behavioral view.
	
		\textcolor{blue}{  \textbf{A1}: We replaced \textit{behaviors view} by \textit{behavior view}. To keep consistency, we also replaced  \textit{features view} by \textit{feature view}}. 

	
	\vspace{3mm}
	\textbf{Q2}: Maybe try to emphasize more already at the early beginning of the manuscript that the feature and the behavioral cluster structure are the same and maybe explain why and when they could be assumed to be aligned; otherwise the improvement in the cold-start problem is unclear.

	\textcolor{blue}{  
	\textbf{A2}: We have added more emphasis on that in the first and second paragraphs of the Introduction. In the second paragraph, we added some more explanation to the scenario of the cold-start problem. 
	} 

	\vspace{3mm}
	\textbf{Q3}: The variables used by Cheng et al. (2014) are unclear.

	\textcolor{blue}{  
	\textbf{A3}: We have specified the type of used variables (page 3).
	} 
	
		
	\vspace{3mm}
	\textbf{Q4}: Bonilla et al (2010) and Abbbasnejad et al (2013) seem to not use dual-view clustering. This should be more explicitly stated and it should be made clearer why they are mentioned, i.e., their relationship to social network clustering.
	
	\textcolor{blue}{  
	\textbf{A4}:  The main difference is actually that they do not use dual-view (i.e.: they do not consider user features). However, they are similar in that users in the same cluster are assumed to have similar behavioral functions. We have clarified this in the paper (page 3, before Section 2)
	} 
	
		\vspace{3mm}
	\textbf{Q5}:  One could state that the generalization to infinite number of clusters leads to a non-parametric model.

	\textcolor{blue}{  
	\textbf{A5}: We had mentioned this fact at the end of the Introduction (page 3). Besides, we have explicitly reminded this again in Section 2.3. We also suggest adding ``non-parametric" to the title of the paper.
	} 

		\vspace{3mm}
	\textbf{Q6}:  Below Equation (1) please correct $p_k$ to $p_K$.

	\textcolor{blue}{  
	\textbf{A6}: We have fixed the subscript from $p_k$ to $p_K$.
	} 

		\vspace{3mm}
	\textbf{Q7}:  The statement "In this case, every component is associated to a cluster." should be toned done as there also exist applications of mixtures where several components are used to form a cluster.
	
	\textcolor{blue}{  
	\textbf{A7}: We have changed it by \textit{is often}.
	} 
	
		\vspace{3mm}
	\textbf{Q8}:  Use consistent bold or non-bold face for $a_u$, $\alpha$ and index $u$.

	\textcolor{blue}{  
	\textbf{A8}: We have changed $a_u$ by $\mathbf{a}_u$ and $\boldsymbol{\alpha}$ by $\alpha$ in equation (3).
	} 
	
		\vspace{3mm}
	\textbf{Q9}: I am not sure if Equation (7) is correct. It gives the impression that this holds for all $k$ where $n_k = 0$, i.e., an infinite amount of k. But rather I would assume that the combined weight of all empty components should be proportional to $\alpha$. This comment also holds for Equation 24. Also Equations 25 and 26 could be modified to avoid the restriction of k within the probabilities.

	\textcolor{blue}{  
	\textbf{A9}: The AE is right. We have fixed the notation in equations (7), (24), (25) and (26).
	} 

	
	\vspace{3mm}
	\textbf{Q10}:  It is unclear why the set $U = \{u_1, \ldots, u_U\}$ is introduced, i.e., why referring to $1, \ldots, U$ is not sufficient. For $\bf{a}_u = (a_1, \ldots, a_D)$ the question arises if a subindex $u$ should also be used for $a_i$, i.e., $a_{ui}$.

	\textcolor{blue}{  
	\textbf{A10}: We have removed the sets $\mathcal{U}$ and $\mathcal{T}$ since they are not used. We have added the subindex $u$ to each element of vector $\mathbf{a}_u$ (page 6) 
	} 
	
	\vspace{3mm}
	\textbf{Q11}:  Below Equation 17 a different notation is used for the precision.

	\textcolor{blue}{  
	\textbf{A11}: We have fixed this typo.
	} 

	
	\vspace{3mm}
	\textbf{Q12}: It would be good to provide some justification, e.g., a reference, for the use of the ML estimators for the priors. One might also explain why the use of these estimates is sufficiently flexible if used as priors for the mixture components.

	\textcolor{blue}{  
	\textbf{A12}: We have no references using this approach.  We used MLE to replicate the idea of center the hyperparameters at the data, applied in the feature view.  We have a added a comment on that in the text (page 8, before Section~3.3).
	} 
	
	\vspace{3mm}
	\textbf{Q13}:  One might explain why choosing m = 3 was assumed to be sufficient.

	\textcolor{blue}{  
	\textbf{A13}: We added some more discussion before Section 4.1 and referred the reader to Neal 2000 for a more systematic discussion. 
	} 

	
	\vspace{3mm}
	\textbf{Q14}:  Add a reference for the iris dataset. Note also that Anderson (1935)'s iris dataset would seem to consist of more observations than used in the example and more variables.

	\textcolor{blue}{  
	\textbf{A14}: We have added a reference to the Anderson (1935), and specified that we use the version available in R as well as the three used features. We have also clarified that we use dropped one of the variables and used a random subset of 50 observations so that the clustering really needs the behavior data to improve the results.  (page 16)
	} 
	
	\vspace{3mm}
	\textbf{Q15}: Maybe explain why different initialization schemes are used for the two experiments and the iris dataset.

	\textcolor{blue}{  
	\textbf{A15}: Since an exhaustive comparison of initialization strategies is out of the scope of the paper, we wanted to at least illustrate that the inference for this model (and the used datasets) is not too sensitive to different initializations. We added some discussion at the end of Section 5.1 (page 11).
	} 

	
	\vspace{3mm}
	\textbf{Q16}:  Maybe modify the figures to not only differentiate groups by color, but also with different plot symbols so that they can also be distinguished when printed in gray scale.

	\textcolor{blue}{  
	\textbf{A16}: We have added different line types and point shapes in the figures.
	} 

	
	\vspace{3mm}
	\textbf{Q17}: Regarding the additional analysis using the iris dataset it would be good to better motivate it. Please specify which observations are included, which variables are included and more clearly explain why it was complemented with artificial data and not a real dataset was used which already contains all required features for applying the proposed model. When evaluating the results it would be good to point out that the class labels are used as ground truth for evaluation while, however, it is unclear in cluster analysis if recovering true class labels is necessarily the aim.

	\textcolor{blue}{  
	\textbf{A17}:  We have specified the three variables used and explained that we use a random subsample of 50 observations so that the clustering really needs the behavior data to improve the results.  (page 17) We have also added some discussion on why we chose the iris dataset and not real forum data (before Section 5.1, page 10). Regarding the ground truth, please note that we do not use the class label to evaluate the results, but the pairwise coincidences (through the ARI index).
	} 
	
	\vspace{3mm}
	\textbf{Q18}: Maybe add round parentheses around vectors, e.g., pi and alpha at the beginning of Appendix A.
	
	\textcolor{blue}{  
	\textbf{A18}: We have added round parentheses around these vectors.
	} 
	
\end{document}

